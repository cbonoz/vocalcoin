\documentclass[conference]{IEEEtran}
  % \documentclass[journal,onecolumn]{IEEEtran}
    % Some Computer Society conferences also require the compsoc mode option,
    % but others use the standard conference format.
    %
    % If IEEEtran.cls has not been installed into the LaTeX system files,
    % manually specify the path to it like:
    % \documentclass[conference]{../sty/IEEEtran}

    \usepackage{chronology}
    \usepackage{graphicx}

    \usepackage{xcolor}

    \newcommand{\cmmnt}[1]{}

    \usepackage{placeins}
    % \usepackage{todonotes}

    % *** CITATION PACKAGES ***
    %
    % \usepackage{cite}
    % \usepackage[style=authortitle -icomp ,natbib=true ,sortcites=true ,block=space]{biblatex} 
    % \documentclass[bibyear]{aa}
      \usepackage{txfonts}
      \usepackage[style=authoryear,sorting=none]{biblatex}
      \usepackage{filecontents}

      \begin{filecontents}{\jobname.bib}
      \end{filecontents}
      \addbibresource{\jobname.bib}

    
    % correct bad hyphenation here
    \hyphenation{op-tical net-works semi-conduc-tor}
    \begin{document}
    %
    % paper title
    % Titles are generally capitalized except for words such as a, an, and, as,
    % at, but, by, for, in, nor, of, on, or, the, to and up, which are usually
    % not capitalized unless they are the first or last word of the title.
    % Linebreaks \\ can be used within to get better formatting as desired.
    % Do not put math or special symbols in the title.
    \title{Vocal Coin\\ A Decentralized Political CryptoCurrency }
    
    % author names and affiliations
    % use a multiple column layout for up to three different
    % affiliations
    \author{
        \IEEEauthorblockN{
            Chris Buonocore \\
        }
        \and
        \IEEEauthorblockN{
            Anup Vasudevan \\
        }
    }

    % make the title area
    \maketitle
    
    % As a general rule, do not put math, special symbols or citations
    % in the abstract
    \begin{abstract}
      The inability for the general public to communicate directly with government officials (and also other community members)is a fundamental roadblock to political effectiveness. Online communication via email and websites like change.org provide one form of an interface to do this, but do not provide a mutual tool for prioritization of policy change. Residents require an easy to use platform to reach their policy makers, and those policy makers need a manageable amount of distinct concerns to satisfy their voters. The Vocal coin system answers two large issues with previous methods of encouraging political action: association of problem topics with the location they are relevant to, and democratic prioritization of important issues. Leveraging Neo to create a proof-of-stake blockchain provides a token economy: participating in a localized issue generates Vocal recognized ether, creating an issue consumes it. Vocal is a platform for mutually incentivized political communication and task prioritization. Vocal is not an investment asset and will not have an ICO, it is rather an accounting and reward system for participants in the Vocal ecosystem - with a custom redemption system that is described later in this paper.
      %% \citation{Voit2012b}
      %% \cite{Voit2012b}
    
    %% \hfill mds
    %% \hfill November, 2017

    \end{abstract}

    % \clearpage
    % no keywords
    
    % For peer review papers, you can put extra information on the cover
    % page as needed:
    % \ifCLASSOPTIONpeerreview
    % \begin{center} \bfseries EDICS Category: 3-BBND \end{center}
    % \fi
    %
    % For peerreview papers, this IEEEtran command inserts a page break and
    % creates the second title. It will be ignored for other modes.
    \IEEEpeerreviewmaketitle
    \section{Introduction}

    Politics has long been considered a field dominated by an older generation, with youth engagement seeming to drop to lower and lower levels.. There are websites currently in place designed to make voting and discovering new political and social issues more user friendly (see change.org), however, these websites tend to suffer from the inability to quickly find issues that are pertinent on a very local and individual level.
    
    Vocal is a political currency platform designed to engage youth in new ways around political topics. Individuals interested in establishing correspondence with any tier of government can earn token by voting on existing issues. For each vote that a user submits, that user is turn rewarded with additional token that can be redeemed for the creation of new issues and/or the promotion of existing ones. To be clear, Vocal is not an investment asset, it is simply an accounting system used for rewarding political involvement.

    There is no Vocal ICO, coin is earned through individual user actions (such as voting) on the Vocalplatform.

%    This provides a rate limit that makes users consider what is genuinely valuable to communicate to their policy makers. There are two details worth investigating:
%  \subsubsection{Bootstrapping} How can the network effect hindering widespread adoption be mitigated?
%  \subsubsection{Crypto Tie In} What utility does the ethereum backed Vocal coin serve?

    % no \IEEEPARstart
    \section{Scope of Use}

    We divide our consideration of the Vocal coin system from the perspective of a few different potential users.

    \subsubsection{Individuals}
    Individuals in the Vocal context are community members interested in public policy affecting public regions. Each resident or community memberis associated with a generated Vocal account. Users may support, reject, or create an issue. Each one of these actions are accepted by the Vocal network, and once made these decisions can't be reversed as the ethereum backed Vocal ledger is immutable. As it is now, it incentivizes research before endorsing an issue, provided this fact is made clear to the resident user.
    Constituents are the ones that introduce Vocal as a platform to their community. The initial aim of Vocal is to grow a geographically clustered user base. This presents two immediate decisions: Vocal must prioritize ease of adoption so that any one who can access the internet has access to this platform, and must target a specific geographic area to bootstrap.
    The first means that Vocal coin must first abstract away all the details in the UI so that people don't have to set up software to communicate with the Neo based network. For now, this accomplishes the usability aspect. Each individual will have a private key, and thus an account, mapped to their platform activity, but it will at first be stored on Vocal's servers, and thus completely transparent. Only when sufficient abstraction exists should this key be exposed by default. This strategy may negate some of the benefits of decentralized consensus, but it's sacrificed temporarily to mitigate the technical on boarding cliff so that Vocal can become a household name for political organization. 

    \subsubsection{Political Constituents}
    Constituents are the ones who may be interested in learning and getting transparency into particular issues and public opinion affecting a particular region. Vocal coin will be later redeemable for access into this information (as well as promotion of their own particular viewpoints and issues). Constituents have the ability to redeem Vocal coin for data and insight into local community issues.

    \subsubsection{Organizations}
    While differing in topology, political organizations have an attribute in common: regional jurisdiction. At first political organizations won't play a large role on the Vocal platform. In the ideal case we'd have some map of jurisidiction as it applies to multiple levels of government. In the United States, there is a hierarchy of governments, with no uniform way of communicating to them and critically identifying who is behind what policy. This weakness is particularly why a cryptocurrency backed ledger is vital: the developers of this platform cannot hope to market its utility effectively to every government system, but once a resident base has formed they can build their own government access protocols on top of the Vocal coin backbone.

    In particular, letting organizations occur as a result of user participation on the network addresses the network problem: Governments will not care about an empty platform. It is much more attractive to link policy makers to an active, vocal community. 

    Open trade is a fundamental aspect of Vocal coin (and financial activity in general). There are a few other efforts which seek to build politically-focused currencies that are publicly trade-able. In contrast, Vocal's model is simpler. Vocal provides a medium for a mutually beneficial relationship between organizations, activists and the general public by providing an exchange of social campaign content that benefits both parties.

    \section{Vocal Coin}

    The Vocal coin is the main unit of exchange and credit in the Vocal ecosystem; however, the utility of Vocal extends to these primary areas:

    The first means that Vocal coin must first abstract away all the details in the UI so that people don't have to set up software to communicate with the Neo based network. For now, this accomplishes the usability aspect. Each individual will have a private key, and thus an account, mapped to their platform activity, but it will at first be stored on Vocal's servers, and thus completely transparent. Only when sufficient abstraction exists should this key be exposed by default. This strategy may negate some of the benefits of decentralized consensus, but it's sacrificed temporarily to mitigate the technical on boarding cliff so that Vocal can become a household name for political organization.

    The two main initial usages of the Vocal coin are the following:

    \begin{enumerate}
      \item As a unit of bookkeeping - tracks for Vocal platform activity and contribution.
      \item As a unit of purchase - the vocal coin can be redeemed for promoting political agenda and gaining additional insight on active and past issues (see the section describing the Constituent user group)
    \end{enumerate}

    \subsection{Blockchain Overview}

    The above model for tracking and transacting Vocal coin implies a large volume of activity (carrying a large amount of additional weight on the network). At the time of this writing, not all of this activity will be able to take place on the Neo main chain (due to known scaling limitations of the Neo blockchain). Additional discussion of the Neo blockchain is out of scope for this draft, but we encourage readers to investigate additional information on both NEP-5 and the Neo blockchain/protocol.

   At it's core, Vocal is a coin which could connect to other blockchains via public exchange - largely backed by the Neo coin. Activity on other blockchains can interlink with this chain. The Vocal chain validates the activity of the behavior of all the participants on the main net (albeit if transactions are performed off the main chain these cannot be tracked). www.LocalBitcoins.com is one example of

    The Vocal coin itself is providing the computation and enforcement of the design via the smart contract protocol. Owning Vocal coin grants the user the right to validate this chain through transaction fees, payment, interchange, trading, and clearinghouse use. 

    \subsection{Holding Vocal Coin}

    \section{Discussion}

    The Vocal architecture can be defined as a simple 3-tier system as illustrated in the below figure.

    \begin{figure}[t]
      \includegraphics[width=8cm]{assets/architecture.png}
      \caption{Basic Vocal Client/Server Architecture}
      \centering
    \end{figure}

     Vocal is deployed as a smart contract on the Neo blockchain.

    \section{Coin Supply}

   Vocal coin follows a fixed coin distribution model as to be described in this section.

    \subsection{Total Cap}
   Vocal will be capped at a total circulating supply of $100,000,000$ $VOCAL$ coins. Coins will be earned by engaging with campaigns (either by interacting, voting or opening campaigns).

     The coin quantity earned for each action will vary according to the total number of supply still remaining. In this sense, voting in campaigns has become the notion of mining - rewarding users for participation on the network. As more coins are earned and held - the coin quantity reward granted will diminish; however, the corresponding market or redeemable value of the coin may correspondingly increase in value. This model lends itself similarly to Bitcoin. In that users will still be given roughly equivalent reward for the amount of work and early holders/participants are rewarded for their early entry point and involvement.

    \section{Distribution}

    The distribution of the coin is initially allocated within a master Neo/Vocal issuer account and will later proceed as follows.

    \begin{enumerate}
        \item Reserved for Vocal Community (60\%)
       Vocal is inherently a publicly traded and earned coin that plays a critical role in the expansion of this new campaign engagement platform. We are fully committed to creating a sound governance structure and we plan to dedicate significant resources to the continued Research and Development of the platform. This allocation will be reserved and distributed as users earn coin by participating in voting on the Vocal web platform.
        \item Retained by Vocal Coin Org for future release (15\%)
        The Vocal coin core development team will be able to sustain itself using funds raised through the coin launch. If the Vocal platform proves itself to be a fundamental technology as we believe it to be, the retained Vocal coins will allow the Vocal development team to sustain operations for many years.
        \item Developer/Community Incentive Fund (15\%)
        The Developer Fund will be used to make targeted capital injections into high potential projects and teams that are attempting to grow the Vocal Coin ecosystem, strategic partnerships, prizes and community development activities.
        \item Founding Team (10\%)
        The founding team’s allocation of Vocal will vest over a traditional 4 year vesting schedule with a one year cliff. This is standard practice for equity vesting and we believe the same standards should be applied to the coin offering.
    \end{enumerate}
    
    % \subsection{Subsection Heading Here}
    % Subsection text here.
    % \subsubsection{Subsubsection Heading Here}
    % Subsubsection text here.


    \subsection{Challenges and Limitations}
    Creating a political and social campaign-based currency poses several challenges. We highlight a few of them here; for example,

    \begin{enumerate}
      \item In order for activists or individuals to participate, the opportunity for them should be equal to or greater than current other opportunities and channels.
      \item Is that true for participation in Vocal? For their participation, they would be both paying for Vocal and be willing to provide a good or service to a consumer in exchange for coins. How is this better than distributing coupon’s from a government organization’s point of view?
      \item Why would an individual want coins? What do they do with them thereafter?
      \item The quality of the social/political campaign view is important to its value. For instance, if a user is voting in campaigns continuously, the value of a single campaign may not be necessarily very high. How does Vocal mitigate this? And how does this affect the reach and importance for individuals or communities who wish to become involved with Vocal Coin?
    \end{enumerate}

    These are fundamental questions to be solved by our Roadmap below.
    
    \section{Development Roadmap - Tentative}

    % https://tex.stackexchange.com/questions/196794/how-can-you-create-a-vertical-timeline

    \newcommand\ytl[2]{
    \parbox[b]{8em}{\hfill{\color{cyan}\bfseries\sffamily #1}~$\cdots\cdots$~}\makebox[0pt][c]{$\bullet$}\vrule\quad \parbox[c]{3cm}{\vspace{7pt}\color{red!40!black!80}\raggedright\sffamily #2.\\[7pt]}\\[-3pt]}
    \begin{table}
    \caption{Vocal Coin Timeline}
    \centering
    \begin{minipage}[t]{\linewidth}
    \color{gray}
    \rule{\linewidth}{1pt}
    \ytl{Feb 2018}{Finish coin proposal and development}
    \ytl{Mar 2018}{Vocal contract deployed - users can create accounts on the Vocal web platform}
    \ytl{Apr 2018}{Continue working on client facing Vocal application}
    \ytl{May 2018}{Vocal application exits beta}
    \ytl{Nov 2018}{Coins redeemable for data and political promotion}
    \ytl{Dev 2018}{Mobile application released for Android}
    \ytl{2019}{TBD: Ongoing Vocal platform security and product development}
    \bigskip
    \rule{\linewidth}{1pt}%
    \end{minipage}%
    \end{table}

    \FloatBarrier
    \section{Conclusion}
   Vocal is a coin that has potential both as a store of social or political campaign value and as an accounting system for political campaign engagement. Taking inspiration from both the Bitcoin and Neo accounting models, one of Vocal's main goals is to create a coin ecosystem whose rewards stay consistent over time (but with greater potential for individual growth with a more early the entry point). The coin is designed to be easily accountable by transactions discoverable on the public Neo blockchain.

    \subsection{Founding Team}
   Vocal was founded by Chris Buonocore and Anup Vasudevan. Chris is a former consultant and past developer from the bay area in California with over 9 years of programming experience.
   He brings wealth of relevant experience to take Vocal to consumers - holding expertise in creating and scaling mobile-based software platforms. Anup is a technical consultant with over 8
   years of development experience as well - primarily web application development and big data pipelines for financial applications.

    \subsection{Neo Vocal Protocol - NEP5}
   Vocal follows the Neo (NEP5-blockchain) token protocol for recording transactions.
    Neo establishes a standard contract AVM for coins on the Neo blockchain and has become the de facto representation for all types of digital assets. Neo coins share the same contract interface, simplifying integration with external contracts.

    Lastly, Vocal coin is an open source project with server and client facing code available on github. View the repository here: https://github.com/cbonoz/vocalcoin.

    % \end{thebibliography}
    \printbibliography
   
    \end{document}
    
    
    